\documentclass[11pt,twoside]{jgsyllabus}\usepackage[]{graphicx}\usepackage[]{xcolor}
% maxwidth is the original width if it is less than linewidth
% otherwise use linewidth (to make sure the graphics do not exceed the margin)
\makeatletter
\def\maxwidth{ %
  \ifdim\Gin@nat@width>\linewidth
    \linewidth
  \else
    \Gin@nat@width
  \fi
}
\makeatother

\definecolor{fgcolor}{rgb}{0.345, 0.345, 0.345}
\newcommand{\hlnum}[1]{\textcolor[rgb]{0.686,0.059,0.569}{#1}}%
\newcommand{\hlstr}[1]{\textcolor[rgb]{0.192,0.494,0.8}{#1}}%
\newcommand{\hlcom}[1]{\textcolor[rgb]{0.678,0.584,0.686}{\textit{#1}}}%
\newcommand{\hlopt}[1]{\textcolor[rgb]{0,0,0}{#1}}%
\newcommand{\hlstd}[1]{\textcolor[rgb]{0.345,0.345,0.345}{#1}}%
\newcommand{\hlkwa}[1]{\textcolor[rgb]{0.161,0.373,0.58}{\textbf{#1}}}%
\newcommand{\hlkwb}[1]{\textcolor[rgb]{0.69,0.353,0.396}{#1}}%
\newcommand{\hlkwc}[1]{\textcolor[rgb]{0.333,0.667,0.333}{#1}}%
\newcommand{\hlkwd}[1]{\textcolor[rgb]{0.737,0.353,0.396}{\textbf{#1}}}%
\let\hlipl\hlkwb

\usepackage{framed}
\makeatletter
\newenvironment{kframe}{%
 \def\at@end@of@kframe{}%
 \ifinner\ifhmode%
  \def\at@end@of@kframe{\end{minipage}}%
  \begin{minipage}{\columnwidth}%
 \fi\fi%
 \def\FrameCommand##1{\hskip\@totalleftmargin \hskip-\fboxsep
 \colorbox{shadecolor}{##1}\hskip-\fboxsep
     % There is no \\@totalrightmargin, so:
     \hskip-\linewidth \hskip-\@totalleftmargin \hskip\columnwidth}%
 \MakeFramed {\advance\hsize-\width
   \@totalleftmargin\z@ \linewidth\hsize
   \@setminipage}}%
 {\par\unskip\endMakeFramed%
 \at@end@of@kframe}
\makeatother

\definecolor{shadecolor}{rgb}{.97, .97, .97}
\definecolor{messagecolor}{rgb}{0, 0, 0}
\definecolor{warningcolor}{rgb}{1, 0, 1}
\definecolor{errorcolor}{rgb}{1, 0, 0}
\newenvironment{knitrout}{}{} % an empty environment to be redefined in TeX

\usepackage{alltt}
\usepackage{booktabs}
%\usepackage{tabularx}
\usepackage{multirow}

\setlength{\aboverulesep}{0.2ex}
\setlength{\belowrulesep}{0.2ex}
\setlength{\extrarowheight}{0.1ex}
\setlength{\heavyrulewidth}{1pt}
\setlength{\lightrulewidth}{0.05pt}
\setlength\emergencystretch{\hsize}

\newif\ifreading
\readingtrue

\iffalse
\newif{\ifisbn}
\isbnfalse

\newcommand{\shortdept}{EES}
\newcommand{\longdept}{EES}
\newcommand{\coursenum}{5891-03}
\newcommand{\sectionnum}{}
\newcommand{\shortcoursetitle}{Bayesian Statistical Methods}
\newcommand{\longcoursetitle}{\shortcoursetitle}
\newcommand{\semester}{Fall}
\newcommand{\yeartaught}{2022}

\newcommand{\Classroom}{Stevenson 6740}
\newcommand{\ClassMeetings}{TR 11:00--12:15\ \Classroom}

\newcommand{\ProfName}{Jonathan Gilligan}
\newcommand{\ShortProfName}{J.\ Gilligan}
\newcommand{\ProfTitle}{Prof.\ Gilligan}
\newcommand{\ProfDescr}{Associate Professor of Earth \& Environmental Sciences\\
Associate Professor of Civil \& Environmental Engineering}
\newcommand{\ProfOffice}{Office: Stevenson 5735 (Stevenson \#5, 7\textsuperscript{th} floor)}
\newcommand{\ProfEmail}{\href{mailto:jonathan.gilligan@vanderbilt.edu}%
{\nolinkurl{jonathan.gilligan@vanderbilt.edu}}}
\newcommand{\ProfWeb}{\href{https://www.jonathangilligan.org}%
{\nolinkurl{www.jonathangilligan.org}}}

\newcommand{\ProfOfficeHours}{Office Hours:
  Monday 2:00--3:00 pm, Tuesday 4:00--5:00 pm,
	or by appointment.%
	} % TODO: Office Hours

\TAfalse
\TaMaletrue
\newcommand{\TaName}{}
\newcommand{\TaTitle}{}
\newcommand{\TaOffice}{}
\newcommand{\TaOfficeLoc}{Office: \TaOffice}
\newcommand{\TaOfficeHours}{Office Hours:
TBA
or by appointment.%
} % Tues. 2:30--3:30}
\TaPhonefalse
\ifTaPhone
\newcommand{\TaPhone}{\qqq}
\fi
\newcommand{\TaEmail}{\href{mailto:bryce.k.belanger@Vanderbilt.Edu}%
{\nolinkurl{bryce.k.belanger@Vanderbilt.Edu}}}
%{\href{qqq}%
%{\nolinkurl{qqq}}}

\newcommand{\FinalExamDay}{Thursday}   %TODO: Final Exam Date & Time
\newcommand{\ShortFinalExamDay}{Thu.{}}   %TODO: Final Exam Date & Time
\newcommand{\FinalExamMonth}{December} %TODO: Final Exam Date & Time
\newcommand{\ShortFinalExamMonth}{Dec.{}} %TODO: Final Exam Date & Time
\newcommand{\FinalExamDate}{15}  %TODO: Final Exam Date & Time
\newcommand{\FinalExamTime}{9:00--11:00~am}  %TODO: Final Exam Date & Time
\newcommand{\FinalExamStartTime}{9:00~am}
\newcommand{\FinalExamEndTime}{11:00~am}
\newcommand{\FinalExamRoom}{\Classroom\ (our regular classroom)}  %TODO: Final Exam Date & Time

\newif\ifAltFinal
\AltFinalfalse

\ifAltFinal
\newcommand{\AltFinalExamDay}{Saturday}   %TODO: Alternate Final Exam Date & Time
\newcommand{\ShortAltFinalExamDay}{Sat.{}}   %TODO: Alternate Final Exam Date & Time
\newcommand{\AltFinalExamMonth}{December} %TODO: Alternate Final Exam Date & Time
\newcommand{\ShortAltFinalExamMonth}{Dec.{}} %TODO: Final Exam Date & Time
\newcommand{\AltFinalExamDate}{10}  %TODO: Alternate Final Exam Date & Time
\newcommand{\AltFinalExamTime}{12:00--2:00~pm}  %TODO: Alternate Final Exam Date & Time
\newcommand{\AltFinalExamStartTime}{12:00~pm}
\newcommand{\AltFinalExamEndTime}{2:00~pm}
\newcommand{\AltFinalExamRoom}{\FinalExamRoom}  %TODO: Alternate Final Exam Date & Time
\fi

\newcommand{\McElreath}{Statistical Rethinking}
\newcommand{\ShortMcElreath}{Rethinking}
\newcommand{\MedMcElreath}{StatisticalRethinking}
\newcommand{\LongMcElreath}{Richard McElreath,
	\emph{Statistical Rethinking: A Bayesian Course with Examples in R and Stan,}
	2\textsuperscript{nd}~ed.{}
	(CRC Press, 2020\ifisbn ; ISBN 978-0-367-13991-9\fi )%
	}
\newcommand{\McElreathWebSite}%
{\href{https://xcelab.net/rm/statistical-rethinking/}%
  {\nolinkurl{xcelab.net/rm/statistical-rethinking/}}%
}%

\newcommand{\Rethinking}{\McElreath}
\newcommand{\ShortRethinking}{\ShortMcElreath}
\newcommand{\MedRethinking}{\MedMcElreath}
\newcommand{\LongRethinking}{\LongMcElreath}
\newcommand{\RethinkingWebSite}{\McElreathWebSite}
\newcommand{\TidyRethinkingBook}%
{\href{https://bookdown.org/content/4857/}%
  {\nolinkurl{bookdown.org/content/4857/}}%
}


\newcommand{\Gomez}{Bayesian Inference with INLA}
\newcommand{\ShortGomez}{INLA}
\newcommand{\MedGomez}{Bayesian Inference with INLA}
\newcommand{\LongGomez}{Virgilio Gomez-Rubio,
	\emph{Bayeian Inference with INLA\/}
	(CRC Press 2021\ifisbn; ISBN 978-1-032-17453-2\fi )%
	}
\newcommand{\GomezURL}%
  {\href{https://becarioprecario.bitbucket.io/inla-gitbook/index.html}%
        {\nolinkurl{becarioprecario.bitbucket.io/inla-gitbook/index.html}}%
  }%
\newcommand{\INLA}{\Gomez}
\newcommand{\ShortINLA}{\ShortGomez}
\newcommand{\MedINLA}{\MedGomez}
\newcommand{\LongINLA}{\LongGomez}
\newcommand{\INLAURL}{\GomezURL}

\newcommand{\Kruschke}{Doing Bayesian Data Analysis}
\newcommand{\ShortKruschke}{DBDA}
\newcommand{\MedKruschke}{Doing Bayesian Data Analysis}
\newcommand{\LongKruschke}{John Kruschke,
    \emph{Doing Bayesian Data Analysis,} 2\textsuperscript{nd}~ed.{}
    (Academic Press, 2015\ifisbn; ISBN 978-0-12-405888-0\fi)%
    }
\newcommand{\DBDA}{\Kruschke}
\newcommand{\ShortDBDA}{\ShortKruschke}
\newcommand{\MedDBDA}{\MedKruschke}
\newcommand{\LongDBDA}{\LongKruschke}

\newcommand{\Wickham}{R for Data Science}
\newcommand{\ShortWickham}{\Wickham}
\newcommand{\LongWickham}{Hadley Wickham and Garrett Grolemund,
  \emph{R for Data Science\/} (O'Reilly, 2017\ifisbn; ISBN 978-1-491-91039-9\fi)%
  }
\newcommand{\WickhamURL}{\href{https://r4ds.had.co.nz/}{\nolinkurl{r4ds.had.co.nz/}}}
\newcommand{\RforDS}{\Wickham}
\newcommand{\ShortRforDS}{\ShortWickham}
\newcommand{\MedRforDS}{\MedWickham}
\newcommand{\LongRforDS}{\LongWickham}
\newcommand{\RforDSURL}{\WickhamURL}

\fi



\SetupHandout
\isbntrue

\renewcommand{\LongCourseName}{\ShortCourseName}%

\title{Syllabus\\
\ShortCourseNumber: \LongCourseName}

\fancyhead[L]{\bfseries\scshape \ShortCourseNumber\ Syllabus}

%
%
%

\IfFileExists{upquote.sty}{\usepackage{upquote}}{}
\begin{document}
\maketitle
\tableofcontents

\clearpage
\section[Nuts \& Bolts]{Nuts and Bolts}
\label{sec:NutsAndBolts}
\subsection{Class Meetings}
\ClassMeetings\\
\subsection{Professor}
\label{sec:Professor}
\ProfName\\
\ProfDescr\\
% \ProfOffice,\\
% \ProfPhone\\
\ProfEmail\\
\ProfWeb\\
\ProfOfficeHours
%
%To book time during my office hours, send me an email
%or make an appointment on line via Google calendar at
%\url{http://bit.ly/o9ka5e}
\iffalse
\subsection{Teaching Assistant}
\label{sec:TA}
\TaName\\
% \TaOfficeLoc\ifTaPhone,\\
% \TaPhone\fi\\
\TaEmail\\
\TaOfficeHours
\medskip

\noindent
\TaTitle\ will be grading labs and homework, so address questions about your
homework and lab grades to \TaAccusative.
\fi

\subsection{Email}
If you want to communicate with Professor Gilligan
% or \TaTitle\
be sure to
begin the subject line of your email with ``{\scshape EES~5891}''
This helps assure
that we will see your message quickly and respond to it.

I have set my email reader to flag all messages like this as important, so I
will read them first.
This also assures that I do not mistake your email for spam. I typically
receive over 100 emails per day, so if you do not follow these instructions,
I may not notice your email.

\subsection{Course web site}
In addition to Brightspace, I have set up a companion web site for this
course at
\url{https://ees5891.jgilligan.org},
where I post the reading and homework assignments,
my slides from class, and other useful material. That web site will be the
central place to keep up with material for the course during the semester.
This web site will direct you to Brightspace if there is anything you need to
find there.

\medskip
\clearpage
\section[Description]{Course Description}
\subsection{Catalog Description}
The class will begin with an introduction to Bayesian statistics and then focus
on practical application of regression methods to data. We will use R together
with the
\href{https://mc-stan.org}{Stan software package} (\url{https://mc-stan.org})
for Hamiltonian Monte Carlo methods
and the \href{https://www.r-inla.org/}{R-INLA software package}
for Integrated Nested Laplace Approximation (INLA) analysis
(\url{https://www.r-inla.org/}).
The course will combine practical applications of Bayesian methods to real
(often messy) data with more philosophical discussions of Bayesian approaches
to statistics and how to interpret results of statistical analyses.
We will focus on regression methods, including hierarchical or
multilevel regression modeling methods, which can be very powerful when you
have data that has a nested structure (e.g., cities and counties within states
or species within genera). Students will do projects applying Bayesian methods
to their own data sets.

\subsection{Prerequisites}
You should be comfortable with differential and integral calculus and have
some previous experience with standard statistics.

This course will be very mathematical and will make extensive use of the R
software system, but I do not assume that you already know R or advanced
mathematics beyond calculus.

\subsection{Narrative Description}
Bayesian statistics is a branch of statistics that has been around for almost
300 years, but for most of that time, it was very difficult to apply to
practical problems because the mathematical equations were too difficult to
solve. In the last 30 years, as computers have become much faster and more
powerful, new computational methods have emerged that make Bayesian statistics
practical for research and applications.

Bayesian analysis is widely used across a wide variety of research as well as
practical applications. It is used to analyze results from high-energy
particle physics experiments to discover new subatomic particles.
There are many other applications in a wide variety of domains.
It's used by geologists to improve estimates of mineral distributions and
radon hazards.
It's used by biologists to identify and categorize variations in the genomes of
humans and other species.
It's used extensively in medicine to analyze the results of clinical trials,
to determine the pharmacokinetics of drug metabolism, and to assess the
predictive value of tests for diseases such as cancer or COVID infection.
It's used in political science and sociology to improve the
accuracy of public opinion surveys and to understand patterns of voting.
It's widely used in marketing to identify consumer preferences and
improve the effectiveness of advertising.
If you use Google, Amazon,
Netflix, Stitchfix, or practically any large online platform for shopping or
entertainment, advanced Bayesian methods form the basis of their
recommendations.
Bayesian analysis has also been applied effectively to law and criminology to
assess the value of evidence in proving guilt or innocence.
It has been applied to public health to estimate the prevalence of dieseases
and tomake more effective treatment decisions when medical tests are uncertain.
It is widely used in meteorology to make weather forecasts and in
climate science to combine data from many different sources and come up with
quantitative predictions and detailed understanding of their associated
uncertainties.
Bayesian methods are also widely used in computational applications, such as
image analysis and reconstruction, computational text analysis, and natural
language processing.
One of the earliest practical applications of Bayesian textual analysis,
in 1964, identified the anonymous authors of the Federalist Papers.
More recent applications of Bayesian textual analysis are used to separate
desired email from spam.

Bayesian statistical methods are valuable because they provide a systematic
way to combine what you already know about a problem with new data from
experiments or observations, and the results of Bayesian analyses are more
straightforward to interpret than conventional statistics.

This course will provide a general introduction to Bayesian statistics and
will combine practical instruction in how to do Bayesian data analysis and
philosophical discussions about how to think about the assumptions that go into
a Bayesian analysis and how to interpret the results that it produces.

You do not need to have any prior knowledge of computer programming, but I do
expect that you are familiar with basic statistics and calculus
(both derivatives and integrals).


\section[Goals]{Goals for the Course}
By the end of the semester, you will:
\begin{itemize}
\item Understand Bayes's theorem and how to apply it.
\item Understand problems with the traditional statistical emphasis on
  null-hypothesis significance testing (NHST), why Bayesian approaches to
  NHST don't solve these problems, and how Bayeian statistics offers
  superior alternatives to NHST.
\item Understand how think about statistical models, how to choose an
  appropriate model for your problems, and understand the tradeoffs between
  different kinds of models.
\item Be able to design and conduct a comprehensive Bayesian analysis of data
  from start to finish.
\item Understand how to choose appropriate priors for your Bayesian analyses
  and how to test whether your choice of priors is sound.
\item Understand how to set up, perform, assess the validity of, and
  interpret the results of Bayesian regression analysis.
\item Understand why Markov Chain Monte Carlo (MCMC) sampling is used in Bayesian
  analysis, what the limits of MCMC are, and how to test your MCMC analyses
  for validity.
\item Understand and be able to perform analyses using more complex
  statistical models, such as interaction models, generalized linear models,
  models of discrete (categorical and count) data.
\item Understand what multilevel or hierarchical models are, when to use them,
  and how to interpret the results of a multilevel analysis.
\item Understand the Integrated Nested Laplace Approximation (INLA),
  why you might use INLA instead of MCMC analysis, and what the limits of INLA
  analysis are.
\item Understand several types of Bayesian geospatial analysis, including
  Matern covariance models and conditional autoregressive (CAR) models.
\end{itemize}

%
%
%
%\clearpage
\iffalse
\section{Important Dates:}
Many of you have athletic and other commitments during the term and may travel
for personal reasons. As you plan for your semester, particularly if you are
purchasing nonrefundable airplane tickets, consult the syllabus.

\fi
%
%
%
%
%
\clearpage
\section[Structure]{Structure of the Course:}
I divide the semester into three parts:
\begin{enumerate}
\item \textbf{Introduction to Bayes's Theorem and its Applications:}
  The first part of the course introduces the basic concepts of Bayesian
  statistics, using simplified approximations to calculate difficult
  equations. This section will focus on linear regression methods.
\item \textbf{Monte Carlo Methods:}
  Next, we study Monte Carlo methods, which help us solve more difficult
  problems that our earlier approximations are not powerful enough for.
  This section will introduce statistical models of discrete data
  (counts, categories, etc.), and generalized linear models. It will conclude
  with multilevel statistical models, which can be very powerful methods for
  working with large and complex data sets.
\item \textbf{Geospatial Modeling:}
  Finally, we will learn a different approach, called the Integrated
  Nested Laplace Approximation (INLA), which is very well suited for
  analyzing geospatial data that may be too difficult to analyze uding
  Monte Carlo methods.
\end{enumerate}

%
%
%
\subsection{Reading Material}
There are two required textbooks and two optional books.
Supplementary reading on the Internet or in handouts will also be assigned
during the term and posted on Brightspace.

\subsubsection[Required Reading]{Required Reading Materials}
\begin{itemize}
	%
	\item \LongRethinking. This will be the principal textbook for the semester.
	  Be sure you get the second edition because it is
	  significantly different from the first.
	\item \LongINLA. We will only use it for a few
	  weeks in the third part of the semester when we study geospatial methods.
	  The book is expensive, but there is a free web-based e-book version
	  online at \INLAURL, which is identical to the printed book.
\end{itemize}
\begin{sloppypar}
There is a companion web site to \emph{\Rethinking\/} at \RethinkingWebSite,
which has links a number of resources, including videos of the author's own
lectures on the material.
\end{sloppypar}

For people who are familiar with R and like to work in the \texttt{tidyverse}
dialect, there is a free companion e-book on the web at \TidyRethinkingBook,
that has translated almost all the R code in the book into the \texttt{tidyverse}
dialect of R.

\subsubsection[Optional Books]{Optional Reading Materials}
\begin{itemize}
  \item \LongDBDA. This book has a more elementary introduction to Bayesian
    statistics, at an undergraduate level. It is very clear, but it
    focuses almost entirely on Monte Carlo sampling and doesn't go as deeply
    into other important aspects of Bayesian statistics as we will do in this
    course.
    It is a very useful resource to check out if Monte Carlo sampling seems
    confusing. I have asked the Science \& Engineering library to put a copy on
    reserve so you can read it there without needing to buy it.
  \item \LongWickham. This book is a great introduction to the R statistical
    programming language. It uses the \texttt{tidyverse} dialect of R,
    developed by Hadley Wickham.
    There is a free web-based ebook version at
    \WickhamURL, so you won't need to buy a paper copy.
\end{itemize}

\subsubsection{Overview of Reading Materials}
I will give out detailed reading that give specific pages to read for each
class and notes on important things you should understand.
\textbf{I expect you to complete the reading before you come to class
% or watch the recorded class session
on the day for which the reading is assigned},
so you can participate in discussions of the assigned material and ask
questions if there are things you don't understand.  %'

%\clearpage
\subsection{Graded Work}
%
%
%
\subsubsection[Grading]{Basis for Grading}

\begin{center}
	\begin{tabular}[t]{cr}
		Class participation & 5\%\\
		Homework & 45\%\\
		Laboratory \& Homework & NA\%\\
		Final exam   & NA\%\\
	\end{tabular}
\end{center}
%
\iffalse
\subsubsection[Participation]{Participation}
To make sure that people who are taking the course asynchronously have lots of
opportunity to participate in discussions, I will be putting discussion
questions on the Brightspace discussion board each week and you will get
participation credit for posting on the discussion boards and responding to
other students' posts. %'

I am also setting up discussion boards for questions about the readings and
lectures and labs. I encourage you to ask on those boards about anything that
you didn't understand or thought was unclear, or would like to see me cover %'
in greater detail in future classes.
\fi
%
%
%
\iftrue
\subsubsection{Homework}
Homework is due at the beginning of class on the day it is assigned.
% Late homework will be accepted for half-credit if I receive it before I post
% the answer key on Brightspace (usually a week after the assignment is due).
\fi
\iftrue
  \subsubsection{Projects}
  You will do an extended research project in the second half of the semester,
  in which you will apply Bayesian methods to investigating a data set that
  you choose. This could be data from your dissertation research or
  another data set that interests you.
\fi

\subsubsection{Tests and Examinations}
\iffalse
\begin{itemize}
  \iffalse
    \item There will be one in-class midterm exam, on
      \textbf{NA}.
      This test will be closed book.
      I will hold a review session before the test.
      \textbf{You will need to bring a calculator, number two pencils,
        and erasers to the in-class test.}
  \else
     \item There will be one take-home midterm exam, which will be due on
       \textbf{NA}.
  \fi
  \item \textbf{Final Examination:}
      The final exam will be take-home and open-book.
      You may use your books and notes.
      You will submit your take-home final electronically on Brightspace.
      The final exam is due by the end of the scheduled
      \ifAltFinal alternate \fi final examination,
      \ifAltFinal
        \AltFinalExamEndTime\ \AltFinalExamDay\
        \AltFinalExamMonth~\AltFinalExamDate
      \else
        \FinalExamEndTime\ \FinalExamDay\ \FinalExamMonth~\FinalExamDate
      \fi.
      The final exam will be cumulative over all the material covered during
      the term.
\end{itemize}
\else
  There will be no tests or exams in this course.
\fi

%\clearpage
\section{Honor Code:}
This course, like all courses at Vanderbilt, is conducted under the Honor Code.
\begin{description}
  \item[Studying:] As you study for this class,
    I encourage you to to seek help from me%
    \ifTA
      , from \TaTitle,
    \else
      \
    \fi
    or from other classmates or friends.
%
  \item[Homework:] I encourage working together.
    In most lab assignments you will be explicitly told to work with a partner.
    I also encourage you to talk with other classmates, as well as friends and
    acquaintances outside of class. You may discuss assignments, compare notes
    on how you are working a problem, and you may look at your
    classmates' work on %'
    homework assignments.
    But you must work through the problems yourself in the work you turn in:
    \textbf{Even if you have discussed the solution with others you must
    work through the steps yourself and express the answers in your own words.
    You may not simply copy someone else's answer.} %'
%
\iffalse
    \item[Tests and Exams:] Tests and exams are different from homework and labs:
    \textbf{all work on tests and exams must be entirely your own}.
    \textbf{You may not work together with anyone or receive any help from
    anyone but me%
    \ifTA
    or \TaTitle
    \fi
    \ on exams and tests (this includes the take-home final exam)}.
\fi
    \item Research project: The research project will be conducted under the
    same ethical principals that apply to publishing papers in scientific
    journals. The work must be your own, but you may consult any other resources.
    If anyone else makes a substantial contribution, you must list them and
    their contributions in an Acknowledgements section.
\end{description}

If you ever
have questions about how the Honor Code applies to your work
in this course, please ask me%
\ifTA
\ or \TaTitle
\fi
.
\textbf{Uncertainty about the Honor Code does not excuse a violation.}

\iffalse
\subsection{Research Integrity}

Beyond the University Honor Code, this course also emphasizes the scientific
ethical principles of research integrity.
Honesty is a very important part of research integrity, but it is only one part.
Clearly, science cannot work if scientists are not scrupulously honest about
the results of their research and there is no tolerance for scientists who
lie. But research integrity goes much farther. Real science happens in the
context of a scientific community and the integrity of this community is
critical. The ethical principles of research
integrity have grown over the centuries to protect the integrity of the
scientific community. Indeed, the mathematician and poet Jacob Bronowski wrote,
in his book, \emph{Science and Human Values\/} (Harper \& Row, 1956) that what
makes science work and makes it great is much less about the intellectual
brilliance and skills of individual scientists than about the ethical commitment
to truth and human dignity by the community of scientists.

Science does not proceed only by making leaps of discovery
but also by making useful mistakes and then discovering and correcting the
errors in those mistakes. Because of this, scientific integrity requires
scientists to be extremely transparent and forthcoming about all the details of
their research. It is not enough to sincerely report a discovery or an idea in
good faith, but one must also provide others with the tools to critically
examine that discovery and idea, and if a scientist learns, even many years
later, that a report or discovery contained an error, they must
correct the error and actively inform other scientists about it.

When a scientist discovers an error in their past work and does not promptly
and actively correct it, other scientists may continue to rely on the truth of
that result and thus waste time, effort, and money.
Thus, both making one's own work available to others so they can have the %'
opportunity to find errors, and also to promptly and publicly report any
errors that one finds in one's own work are two critical pieces of research %'
integrity.

Another aspect also involves the communal nature of science: None of us works
in isolation, and every scientist's work builds on work by others. %'
There are two reasons why it is critical to acknowledge the role of others' %'
work in our own research reports: First, it is important to give others the
credit for their contributions to our shared body of scientific knowledge.
Secondly, it is important for others to know where the data and methods we use
come from.
If I use someone else's data or methods for an analysis and it later turns %'
out that there were problems with their data or methods, then it is important
for people reading my work to be able to examine my work and evaluate how those
errors might affect my own results.

I want to emphasize that these considerations about research integrity are not
just negative things. They are very positive, which is why so many researchers
are embracing them.
\textbf{By being transparent and forthcoming, and by encouraging
others to reproduce your research results, you can enhance your reputation},
both for honesty (you show that you have nothing to hide) and for being a
good citizen of the scientific community by making it easy for other researchers
to learn from your work and build on it to make new discoveries and build
new and more powerful tools for analyzing data.

Where this is relevant to this course on Global Climate Change is in our
practice of reproducible research in the laboratory portion of the course.
Making our work, however humble, fully open and transparent so that others
may examine it, criticize it, or build on it to develop new tools and make
new discoveries is an essential part of research integrity.

In your lab reports it will be important for you to document where the data you
worked with comes from (this will mostly be clearly spelled out in the
assignments) and what methods you used to analyze it. Using the tools of R and
RMarkdown will make it easy to almost automatically include this kind of
transparency in your reports. As you do this throughout this course, you will
learn the best practices adopted by the scientific community and develop habits
of openness, transparency, and reproducibility for any research you do in the
future in any area of society, whether in science, journalism, business, or
other endeavors.
\fi

\section{Final Note:}
I have made every effort to plan a busy, exciting, and instructive semester.
I may find during the term that I need to revise the syllabus to give more time
to some subjects or to pass more quickly over others rather than covering them
in depth. Many topics we will cover are frequently in the news. Breaking news
may warrant a detour from the schedule presented on the following pages.
Thus, while I will attempt to follow this syllabus as closely as I can,
you should realize that it is subject to change during the semester.

\clearpage
\section{Meet Your Professor}
Jonathan Gilligan has worked in many areas of science and public policy.
His past research includes work on laser physics, quantum optics,
laser surgery, electrical properties of the heart, using modified spy planes to
study the ozone layer in the stratosphere, and connections between religion and
care for the environment.
\iffalse

Professor Gilligan joined the Vanderbilt Faculty in 1994 as a member of the
Department of Physics and Astronomy. In 2003, when the Department of Geology
became the Department of Earth and Environmental Science, Professor Gilligan
joined the new department to focus on atmospheric science, global climate change,
and the interactions of politics, ethics, religion, communication, and the
environment.
\fi

Professor Gilligan is
the Alexander Heard Distinguished Service Professor,
Associate Professor of Earth \& Environmental Sciences,
Associate Professor of Civil \& Environmental Engineering, and
the director of the Vanderbilt Climate and Society Grand Challenge Initiative,
which is working to integrate research, teaching, and public outreach about
climate change across the natural sciences, social sciences, and humanities.

His current research investigates the role of individual and household behavior
in greenhouse gas emissions in the United States;
how ``smart cities'' can use technology to reduce environmental footprints and
promote health and citizen empowerment;
water conservation policies in American cities;
vulnerability and resilience to environmental stress in South Asia;
and developing new directions for climate policy in the US.

Professor Gilligan and Professor Michael Vandenbergh won
the 2017 Morrison Prize for the highest-impact paper of the year
on sustainability law and policy.
Gilligan and Vandenbergh's book, %'
\emph{Beyond Politics: The Private Governance Approach to Climate Change\/}
(Cambridge University Press, 2017),
was named by \emph{Environmental Forum\/} as one of the most
important books on the environment of the last 50 years.

Apart from his academic work, Professor Gilligan dabbles in writing for the
theater. His stage adaptation of Nathaniel Hawthorne's %'
\emph{The Scarlet Letter},
co-written with his mother Carol Gilligan, has been staged at The Culture
Project in New York City, starring
Marisa Tomei, Ron Cephas Jones, and Bobby Cannavale, and was later performed
at Prime Stage Theatre, Pittsburgh and in a touring production by The National
Players. Most recently, it was performed as the principal fall 2019 production
of the Fullerton College Classic Dramatic Series in Fullerton CA,
directed by Michael Mueller,
and was also chosen by the Classic Repertory Company in Watertown, MA,
for its 2019--2020 repertory season.

Prof.\ Gilligan and Carol Gilligan also wrote the libretto for an opera,
\emph{Pearl}, in collaboration composer Amy Scurria, and producer/conductor
Sara Jobin, which was performed at Shakespeare \& Company in Lenox MA,
starring Maureen O'Flynn, John Bellemer, Marnie Breckenridge, John Cheek, %'
and Michael Corvino, and in Shanghai China,
% as part of a cultural exchange,
starring Li Xin, Wang Yang, John Bellemer, and Lin Shu.
%
%
%
%
%
%
%\end{document}

\clearpage
\cleardoublepage
\appendix
\setcounter{secnumdepth}{0}
\newcommand{\maybehline}{\hline}%
\setlength\extrarowheight{4pt}
\section[Class Schedule]{Schedule of Classes
\ifrevised
	(Revised \RevisionDate)%
\else
	(Subject to Change)%
\fi}

\textbf{\scshape Important Note:} This schedule gives a rough indication of the
reading for each day. See the detailed daily assignments on the course web
site at \url{https://ees5891.jgilligan.org}.

\setlength{\aboverulesep}{0.2ex}
\setlength{\belowrulesep}{0.2ex}
\setlength{\extrarowheight}{0.1ex}
\setlength{\heavyrulewidth}{0.5pt}
\setlength{\lightrulewidth}{0.05pt}
\begin{center}
% latex table generated in R 4.2.1 by xtable 1.8-4 package
% Mon Aug 29 14:45:03 2022
\begin{tabular}{l@{~}c@{~}r>{\raggedright}m{2.6in}>{\centering}m{2in}c}
  \toprule
  \multicolumn{3}{c}{\bfseries Date} & \multicolumn{1}{c}{\bfseries Topic} &\multicolumn{1}{c}{\bfseries Reading} &\\
 \midrule
Thu., & Aug. &  25 & Introduction & No reading &  \\ 
   \midrule
Tue., & Aug. &  30 & Rethinking statistics & \emph{McElreath\/} Ch.~1--2 ("The Golem of Prague" and "Small Worlds and Large Worlds") &  \\ 
   \midrule
Thu., & Sep. &   1 & Sampling & \emph{McElreath\/} Ch.~3 ("Sampling the Imaginary") &  \\ 
   \midrule
Tue., & Sep. &   6 & Geocentric Models & \emph{McElreath\/} Ch.~4 ("Geocentric Models") &  \\ 
   \midrule
Thu., & Sep. &   8 & Many variables & \emph{McElreath\/} Ch.~5 ("The Many Variables and The Superfluous Waffles") &  \\ 
   \midrule
Tue., & Sep. &  13 & Designing statistical models & \emph{McElreath\/} Ch.~6 ("The Haunted DAG \& The Causal Terror") &  \\ 
   \midrule
Thu., & Sep. &  15 & Regularization & \emph{McElreath\/} Ch.~7 ("Ulysses' Compass") &  \\ 
   \midrule
Tue., & Sep. &  20 & Interactions & \emph{McElreath\/} Ch.~8 ("Conditional Manatees") &  \\ 
   \midrule
Thu., & Sep. &  22 & Monte Carlo sampling & \emph{McElreath\/} Ch.~9 ("Markov Chain Monte Carlo") &  \\ 
   \midrule
Tue., & Sep. &  27 & Generalized linear models & \emph{McElreath\/} Ch.~9 ("Markov Chain Monte Carlo") &  \\ 
   \midrule
Thu., & Sep. &  29 & Generalized linear models & \emph{McElreath\/} Ch.~10 ("Big Entropy and the Generalized Linear Model") &  \\ 
   \midrule
Tue., & Oct. &   4 & Discrete statistical models & \emph{McElreath\/} Ch.~11 ("God Spiked the Integers") &  \\ 
   \midrule
Thu., & Oct. &   6 & Mixture models & \emph{McElreath\/} Ch.~12 ("Monsters and Mixtures") &  \\ 
   \midrule
Tue., & Oct. &  11 & Discussion of student projects & No reading &  \\ 
   \midrule
Thu., & Oct. &  13 & Fall Break & No reading &  \\ 
   \bottomrule
\end{tabular}

\end{center}

\begin{center}
% latex table generated in R 4.2.1 by xtable 1.8-4 package
% Mon Aug 29 14:45:03 2022
\begin{tabular}{l@{~}c@{~}r>{\raggedright}m{2.6in}>{\centering}m{2in}c}
  \toprule
  \multicolumn{3}{c}{\bfseries Date} & \multicolumn{1}{c}{\bfseries Topic} &\multicolumn{1}{c}{\bfseries Reading} &\\
 \midrule
Tue., & Oct. &  18 & Multilevel models & \emph{McElreath\/} Ch.~13 ("Models with Memory") &  \\ 
   \midrule
Thu., & Oct. &  20 & Multilevel models, part 2 & \emph{McElreath\/} Ch.~13 ("Models with Memory") &  \\ 
   \midrule
Tue., & Oct. &  25 & More multilevel models & \emph{McElreath\/} Ch.~14 ("Adventures in Covariance") &  \\ 
   \midrule
Thu., & Oct. &  27 & Messy data & \emph{McElreath\/} Ch.~15 ("Missing Data and Other Opportunities") &  \\ 
   \midrule
Tue., & Nov. &   1 & Geospatial data analysis & \emph{Gomez-Rubio\/} &  \\ 
   \midrule
Thu., & Nov. &   3 & Laplace approximations & \emph{Gomez-Rubio\/} &  \\ 
   \midrule
Tue., & Nov. &   8 & Nested Laplace approximations & \emph{Gomez-Rubio\/} &  \\ 
   \midrule
Thu., & Nov. &  10 & Matern models & \emph{Gomez-Rubio\/} &  \\ 
   \midrule
Tue., & Nov. &  15 & Conditional autoregressive models & \emph{Gomez-Rubio\/} &  \\ 
   \midrule
Thu., & Nov. &  17 & STAN Hamiltonian Monte Carlo sampler & No reading &  \\ 
   \midrule
Tue., & Nov. &  22 & \multicolumn{2}{l}{\multirow{3}{*}{\bfseries\scshape\Large Thanksgiving Break}} & \\% & No reading &  \\ 
  Thu., & Nov. &  24 &  & No reading &  \\ 
   \midrule
Thu., & Dec. &   1 & Diagnosing model output & No reading &  \\ 
   \midrule
Tue., & Dec. &   6 & Project presentations & No reading &  \\ 
   \midrule
Thu., & Dec. &   8 & Project presentations & No reading &  \\ 
   \midrule
Sat., & Dec. &  10 &  & No reading &  \\ 
   \bottomrule
\end{tabular}

\end{center}
%
\end{document}
